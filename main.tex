\documentclass{beamer}
%
% Choose how your presentation looks.
%
% For more themes, color themes and font themes, see:
% http://deic.uab.es/~iblanes/beamer_gallery/index_by_theme.html
%

\mode<presentation>

  \usetheme{Warsaw}      % or try Darmstadt, Madrid, Warsaw, ...\textbf{\(\(\(\)\)\)}}

  \usecolortheme{default} % or try albatross, beaver, crane, ...
  \usefonttheme{default}  % or try serif, structurebold, ...
  \setbeamertemplate{navigation symbols}{}
  \setbeamertemplate{caption}[numbered]
\usepackage{ragged2e}
\usepackage[english]{babel}
\usepackage[utf8x]{inputenc}

\title[Pêndulo Magnético]{Pêndulo Magnético}
\author{PSP}
\date{Abril de 2015}

\begin{document}

\begin{frame}
  \titlepage
\end{frame}

\begin{frame}{Pêndulo Magnético}
\justifying
Faça um pêndulo leve com um pequeno ímã em sua extremidade. Um eletroímã adjacente conectado a uma fonte de corrente alternada de frequência muito superior à frequência natural do pêndulo pode levar a oscilações não-amortecidas com várias amplitudes. Estude e explique o fenômeno.
\end{frame}

\begin{frame}{Pêndulo Magnético}
  \tableofcontents
\end{frame}

\section{Introdução Teórica}

\subsection{Visão Geral}
\begin{frame}{Visão Geral}
\begin{itemize}
\item O pêndulo consiste em um braço rígido fixado em um pivô com pouco atrito, construído de forma a se mover em um plano horizontal, com um pequeno imã preso à sua extremidade livre. 
\item Um eletroímã é colocado logo abaixo do ponto mais baixo do movimento do pêndulo, de tal forma que o eletroímã exerce uma aceleração ou desaceleração no pêndulo, que depende da polaridade da corrente e da direção do imã.
\item O campo de ação do eletroímã no pêndulo é escolhido curto, de tal forma que a ação do eletroímã apenas se torna significativa dentro deste campo.
\end{itemize}
\end{frame}

\begin{frame}{Visão Geral}
\begin{itemize}
\item Logo depois, conectamos o eletroímã a uma fonte senoidal de corrente alternada, cuja frequência $\omega_{campo}$ e voltagem podem ser variadas em um largo domínio (cerca de 10 a mil vezes a frequência natural do pêndulo $f_0$).
\item Quando a voltagem é tal que a interação entre o pêndulo e o eletroímã é significativa, é possível observar alguns efeitos.
\end{itemize}
\end{frame}

\begin{frame}{Propriedades do Pêndulo Magnético}
\begin{itemize}
\item Quando solto de uma posição qualquer, o movimento do pêndulo vai se aproximando do regime estacionário de um pêndulo simples sem atrito.
\item Estes regimes quase estacionários ocorrem apenas quando o pêndulo atinge uma das "amplitudes permitidas". 
\item O período do pêndulo em um regime estável é próximo ao período de um pêndulo simples.
\item A perda de energia por dissipação é compensada pelo pêndulo em uma espécie de auto-regulação.
\end{itemize}
\end{frame}

\begin{frame}{Propriedades do Pêndulo Magnético}
\begin{itemize}
\item A "escolha" do sistema por um modo estável depende das condições iniciais.
\item As amplitudes quantizadas não dependem da intensidade do campo. (Desde que esta seja suficientemente grande para haver uma interação significativa eletroímã-pêndulo).
\item Quando o campo não é forte o suficiente para suportar as perdas por atrito em um modo estável, o pêndulo "salta" para uma amplitude permitida menor (logo, de menor energia).
\end{itemize}
\end{frame}
\subsection{Equação de Movimento}
\begin{frame}{Equação de Movimento}
\begin{itemize}
\item Pela 2ª lei de Newton, temos que: $$ ml\ddot{\phi} =  -mg\sin \phi - \beta m l \dot{\phi} + m l A \epsilon(\phi) \sin(\omega_{campo} t)   $$ 
\item Dividindo isto por $ml$, chegamos à nossa equação de movimento:
\begin{equation}\label{eq_mov}
\ddot{\phi} + \beta \dot{\phi} + \omega_0^2 \sin \phi  =  A \epsilon(\phi) \sin(\omega_{campo} t) 
\end{equation}
\end{itemize}
\end{frame}

\begin{frame}{Dois tipos de oscilação}
\begin{itemize}
\item Há dois tipos de oscilação possíveis para o pêndulo:
\item Quando o pêndulo tem seu movimento totalmente dentro do campo de ação do imã, seu movimento descrito por \eqref{eq_mov} se reduz ao movimento de um oscilador forçado e amortecido: $$ \ddot{\phi} + \beta \dot{\phi} + \omega_0^2 \phi = A \sin(\omega_{campo} t) $$
\item Mas o problema solicita o estudo das oscilações não amortecidas, que é o nosso segundo tipo de oscilação, na qual o movimento do imã não está completamente dentro da área de ação do campo magnético.
\end{itemize}
\end{frame}

\subsection{Modulação da velocidade}
\begin{frame}{Modulação da velocidade}
\begin{itemize}
\item Se o campo completa um número inteiro de ciclos durante a passagem do pêndulo pela zona de contato, os efeitos se cancelam e não há ganho ou perda de energia. Se isto não acontece, há uma tranferência de energia do campo para o pêndulo.
\item Sinal e magnitude da energia transferida para o pêndulo depende também do "tempo de vôo". Porém, sempre há uma fase de entrada que possibilita o ganho de energia,independente da velocidade.
\item Sistema não linear e com muitos vínculos complicados. A melhor saída é observar seu comportamento experimental.
\end{itemize}
\end{frame}

\subsection{Quantização das Amplitudes}
\begin{frame}{Quantização das Amplitudes}
\begin{itemize}
\item É observado que acoplamento do pêndulo com o campo magnético resulta em um sistema oscilatório com um espectro discreto de regimes estáveis.
\item Cada um desses regimes está associado a uma determinada amplitude de uma oscilação em regime quase estacionário.
\item Em cada um destes regimes o pêndulo oscila com frequência próxima à sua frequência natual.
\end{itemize}
\end{frame}

\begin{frame}{Quantização das Amplitudes}
\begin{itemize}
\item Quando o pêndulo está em um regime quase estacionário, sua perda de energia é constante durante meia oscilação. Portanto, a energia recebida em cada meio período tem que ser igual. 
\item A transferência de energia depende da fase do campo quando há interação com o pêndulo. Esta fase está relacionada com o período do pêndulo.
\item Em um pêndulo de amplitude qualquer, para cada amplitude, há um determinado período.
\end{itemize}
\end{frame}

\begin{frame}{Quantização das Amplitudes}
\begin{itemize}
\item A cada meio período, o pêndulo deve receber a mesma quantidade de energia do campo magnético.  
\item Como a velocidade do magneto a cada duas passagens sucessivas pelo ponto mais baixo é invertida, concluímos que a fase do campo deve ser a oposta da oscilação anterior. 
\item  $\frac{T}{2} = (m + \frac{1}{2})t \rightarrow T = nt$
\end{itemize}
\end{frame}

\begin{frame}{Quantização das Amplitudes}
\begin{itemize}
\item Período clássico para uma amplitude qualquer:$$T = \frac{1}{f_0}\left(1 + \frac{\theta_0^2}{16}\right)$$ 
\item Usando a relação entre os períodos: $$ \theta_0 =  4\sqrt{\frac{n f_0}{f_{campo}} - 1} \approx 4\sqrt{1-\frac{f_{campo}}{n f_0}}$$
\end{itemize}
\end{frame}

\subsection{Análise dos resultados}
\begin{frame}{Análise dos resultados}
\begin{itemize}
\item As amplitudes calculadas teoricamente mostram apenas uma correspondência aproximada para as "amplitudes quantizadas" observadas computacionalmente.
\item Efeitos de atrito e mudanças na velocidade dentro da zona de interação.
\item Quanto maior a amplitude, menor a discrepância.
\end{itemize}
\end{frame}

\section{Computacional}

\subsection{Computacional}

\begin{frame}{Computacional}

\begin{itemize}
\item bla bla
\end{itemize}

\end{frame}


\end{document}
