\documentclass{beamer}
%
% Choose how your presentation looks.
%
% For more themes, color themes and font themes, see:
% http://deic.uab.es/~iblanes/beamer_gallery/index_by_theme.html
%

\mode<presentation>

  \usetheme{Warsaw}      % or try Darmstadt, Madrid, Warsaw, ...\textbf{\(\(\(\)\)\)}}

  \usecolortheme{default} % or try albatross, beaver, crane, ...
  \usefonttheme{default}  % or try serif, structurebold, ...
  \setbeamertemplate{navigation symbols}{}
  \setbeamertemplate{caption}[numbered]
\usepackage{ragged2e}
\usepackage[english]{babel}
\usepackage[utf8x]{inputenc}

\title[Portafolio ETFs]{Portafolio ETFs}
\subtitle[Portafolio ETFs]{PAP Conformación de portafolios de inversión para economías
familiares y desarrollo de alternativas de micro inversión eficientes}
\author{Luisa Fernanda Aguilar}
\date{Octubre de 2017}

\begin{document}

\begin{frame}
  \titlepage
\end{frame}

\begin{frame}{Portafolio ETFs}
\justifying
A continuación se presenta la metodología para la creación de un portafolio conformado con ETFs, a lo largo del semestre Otoño 2017 en el Proyecto de Apreciación Profesional de Conformación de portafolios de inversión para economías familiares y desarrollo de alternativas de micro inversión eficientes.
\end{frame}

\begin{frame}{Portafolio ETFs}
  \tableofcontents
\end{frame}

\section{Introducción}

\subsection{ETFs}
\begin{frame}{¿Qué es un ETF?}
\justifying
Un fondo cotizado (ETF por sus siglas en inglés) es un conjunto diversificado de activos(como un fondo de inversión) que cotiza en bolsa (como una acción). Los ETFS son una forma sencilla, económica y eficiente en costes de invertir. 
\end{frame}

\begin{frame}{Ventajas de Invertir en ETFs}
\justifying
El precio, la comodidad y la variedad son algunas de las razones por las que los inversores deberían considerar utilizar fondos cotizados (ETFs) en sus carteras.
\begin{itemize}
\item \textbf{ Flexibilidad:} La variedad de exposiciones disponibles hacen de los ETFS un instrumento válido tanto para el núcleo de la cartera como para inversiones más tácticas.
\item \textbf{Eficientes en costes:} Los ETFs suelen tener menos comisiones de gestión que los fondos y los costes son fáciles de calcular.
\item \textbf{Diversificación:} Los ETFs ofrecen de forma instantánea una fuente de exposición a la rentabilidad de distintos valores.
\item \textbf{Transparencia:} Los ETFs muestran cada valor contenido en el fondo para que sepa exactamente en lo que está invertido.
\end{itemize}
\end{frame}

\begin{frame}
\begin{itemize}
\item \textbf{Liquidez:} Los ETFs cotizan en bolsa y pueden negociarse en cualquier momento durante el horario de mercado.
\item \textbf{Acceso:} Los ETFs permiten acceso inmediato a mercados internacionales.
\end{itemize}
\justify 
En la actualidad, el sector de ETFs mueve 4 billones de dólares estadounidenses
\end{frame}


\begin{frame}{Cómo elegir un ETF}
\begin{itemize}
\item Aumentar el patrimonio a largo plazo
\item Generar rentas
\item Beneficiarse de los movimientos a corto plazo
\item \textbf{Exposición:} Encuentre un ETF que replique un índice para reflejar su estrategia de inversión
\item \textbf{Precio:} ¿Cuánto está dispuesto a pagar para replicar el índice que ha seleccionado?
\item \textbf{Estructura:} Asegúrate de conocer la estructura de tu ETF que puede ser de replica física o sintética
\end{itemize}
\end{frame}

\begin{frame}{Los tres grandes}
\justify 
El mercado global de ETFs está dominado por tres grandes compañías: BlackRock Inc.'s (NYSE: BLK
BlackRock Inc BLK 462.45-0.17\%) iShares, Vanguard's ETFs y State Street Global Advisors’ (SSgA) SPDR ETFs. Para Mayo de 2016, alrededor del 70\% del mercado global de ETF y activos ETP eran administrados por estas compañías. Ningun otro proveedor llega si quiera al 3.5\% del mercado global. 
\end{frame}

\section{Metodología}

\subsection{Metodología}

\begin{frame}{Metodología}

\begin{itemize}
\item bla bla
\end{itemize}

\end{frame}


\end{document}
